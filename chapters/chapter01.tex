\chapter{2025年3月}
\section{3月第1次测试}
\section*{函数单调性、定积分换元法结合函数零点[1-3题]}
%1
\begin{question}
    在  $(0,+\infty)$  内,方程  $x^{2} \int_{0}^{1} \sqrt{1-x^{2}} \mathrm{~d} x-2 \mathrm{e} \ln x=0$的实根个数为(\qquad)
  
    \begin{choices}
      \item $0$
      \item $1$
      \item $2$
      \item $3$
    \end{choices}
  \end{question}
\begin{solution}
    \[\int_{0}^{1} \sqrt{1-x^2} \,dx \stackrel{x = sint}{=} \int_{0}^{\frac{\pi}{2}} {cos(t)}^2 \,dt = \frac{\pi}{4}\]
    令 \( f(x) = \frac{\pi}{4}x^2 - 2e lnx \) ,则 \( f\prime(x) = \frac{ \pi(x - \sqrt{\frac{4e}{\pi}})(x + \sqrt{\frac{4e}{\pi}})}{2x} \)\\
    令\(f\prime(x) = 0\)得,\(x_1 = \sqrt{\frac{4e}{\pi}}, x_2 = -\sqrt{\frac{4e}{\pi}}\)\\
    当\(x \in (0,x1)\)时,\(f\prime(x) < 0\),故\(f(x)\)单调递减\\
    当\(x \in (x_1,+\infty)\)时,\(f\prime(x) > 0\), 故\(f(x)\)单调递增\\
    而\(\lim_{x \to 0^+} f(x) = +\infty\), \(\lim_{x \to +\infty} f(x) = +\infty\)\\
    并且 \(f(x_1) = e ln(\frac{\pi}{4}) < 0\), 所以此方程有两个实根
\end{solution}
%1题补充习题1v
\begin{question}
    \textbf{[第1题补充习题01]}
    在\((0,+\infty)\)内,方程\(x^{2}\int_{0}^{1}\sqrt{4 - x^{2}}dx - 3e^{2}\ln x = 0\)的实根个数为(\qquad )
    \begin{choices}
        \item \(0\)
        \item \(1\)
        \item \(2\)
        \item \(3\)
    \end{choices}
\end{question}
\begin{solution}
    令\(x = 2sint\), 则\(dx = 2costdt\)。\\
    当\(x = 0\)时, \(t = 0\); 当\(x = 1\)时, \(sint = \frac{1}{2}\), 即\(t = \frac{\pi}{6}\)。\\
    那么\(\int_{0}^{1} \sqrt{4 - x^2} = \int_{0}^{\frac{\pi}{6}} 2cost \cdot 2cost \,dt = 4\int_{0}^{\frac{\pi}{6}} {cost}^2 \)\\
    根据三角函数的二倍角公式\({cost}^2 = \frac{1+cos2t}{2} \,dt\),可得:
    \begin{align*}
        &4\int_{0}^{\frac{\pi}{6}} {cost}^2 \,dt = 4\int_{0}^{\frac{\pi}{6}} \frac{1+cos2t}{2} \,dt = 2\int_{0}^{\frac{\pi}{6}} (1+cos2t) \,dt\\
        &=2\left(t + \frac{1}{2}sin2t\right)\bigg|_{0}^{\frac{\pi}{6}} = 2(\frac{\pi}{6} + \frac{\sqrt{3}}{4}) = \frac{\pi}{3} + \frac{\sqrt{3}}{2}
    \end{align*}
    令\(f(x) = \left(\frac{\pi}{3} + \frac{\sqrt{3}}{2}\right)x^2 - 3{e}^2lnx ,x \in (0, + \infty)\)\\
    则\(f\prime(x) = 2\left( \frac{\pi}{3} + \frac{\sqrt{3}}{2}\right)x - \frac{3{e}^2}{x} = \frac{2\left( \frac{\pi}{3} + \sqrt{3}\right)x^2 - 3{e}^2}{x}\)\\
    令\(f\prime(x) = 0\), 即\(2\left( \frac{\pi}{3} + \sqrt{3}\right)x^2 - 3{e}^2 = 0\)\\
    解得\(x_1 = \sqrt{\frac{3{e}^2}{2(\frac{\pi}{3} + \frac{\sqrt{3}}{2})}},x_2 = -\sqrt{\frac{3{e}^2}{2\left(\frac{\pi}{3} + \frac{\sqrt{3}}{2}\right)}} \text{($x >0,$舍去负根)}\)\\
    %单调性分析
    当\(0 < x < x_1\)时,\(f\prime(x) < 0\),所以\(f(x)\)在\((0, x1)\)上单调递减;\\
    当\(x > x_1\)时,\(f\prime(x) > 0\),所以\(f(x)\)在\((x_1, +\infty)\)上单调递增。\\
    %分析函数极限和极值
    而
    \[
    \lim_{x \to 0^+} f(x) = \lim_{x \to 0^+} (\frac{\pi}{3} + \frac{\sqrt{3}}{2})x^2 - 3e^2lnx = +\infty,
    \]
    \[
        \lim_{x \to +\infty} f(x) = \lim_{x\to+\infty} (\frac{\pi}{3} + \frac{\sqrt{3}}{2})x^2 - 3e^2lnx = +\infty
    \]
    \(f(x)\)在\(x = x_1\)处取得极小值,也是最小值
    \begin{align*}
        &f(x_1) = (\frac{\pi}{3} + \frac{\sqrt{3}}{2})x_1 - 3e^2lnx_1 \\
        &= \frac{3e^2}{2} ln\left(\frac{2\pi + 3\sqrt{3}}{9e}\right)<0
    \end{align*}
    故函数\( y =f(x)\)的图像与\(x\)轴有两个交点,即方程\(x^2\int_{0}^{1} \sqrt{4 - x^2} \,dx-3e^2lnx = 0\text{在}(0,+\infty)\)内的实根个数为2。
\end{solution}
%第一题补充习题2
\begin{question}
 \textbf{[第1题补充习题02]}  
 设 $f(x) = x^3 \int_0^2 \sqrt{4 - t^2} \, \mathrm{d}t - 4e \ln x$,则 $f(x)$ 在 $(0, +\infty)$ 内零点个数为(\qquad)
 \begin{choices}
    \item \(0\)
    \item \(1\)
    \item \(2\)
    \item \(3\)
 \end{choices}
\end{question}
\begin{solution}
    该题可先计算定积分\(\int_{0}^{2}\sqrt{4 - t^{2}}dt\)的值,再对函数\(f(x)\)求导,分析其单调性、极值和极限情况,从而确定\(f(x)\)在\((0, +\infty)\)内的零点个数。

    %步骤一:计算定积分\(\int_{0}^{2}\sqrt{4 - t^{2}}dt\)**
    令\(t = 2\sin u\),则\(dt = 2\cos udu\)。
    当\(t = 0\)时,\(u = 0\);当\(t = 2\)时,\(\sin u = 1\),\(u = \frac{\pi}{2}\)。
    \begin{align*}
    & \int_{0}^{2}\sqrt{4 - t^{2}} \mathrm{d}  \int_{0}^{\frac{\pi}{2}}\sqrt{4 - 4\sin^{2}u}\cdot 2\cos udu=\int_{0}^{\frac{\pi}{2}}2\cos u\cdot 2\cos udu\\
    &= 4\int_{0}^{\frac{\pi}{2}}\cos^{2}udu= 4\int_{0}^{\frac{\pi}{2}}\frac{1 + \cos2u}{2}du\\
    &= 2\int_{0}^{\frac{\pi}{2}}(1 + \cos2u)du= 2\left(u + \frac{1}{2}\sin2u\right)\Bigg|_{0}^{\frac{\pi}{2}}\\
    &= 2\left(\frac{\pi}{2} + \frac{1}{2}\sin\pi - 0 - \frac{1}{2}\sin0\right)\\
    &=\pi
    \end{align*}

    %步骤二:确定函数\(f(x)\)的表达式
    将定积分的值代入\(f(x)\),可得\(f(x)=\pi x^{3}-4e\ln x\),\(x\in(0, +\infty)\)。

    %步骤三:求\(f(x)\)的导数并分析单调性
    对\(f(x)\)求导:\(f^\prime(x)=3\pi x^{2}-\frac{4e}{x}=\frac{3\pi x^{3}-4e}{x}\)。\\
    令\(f^\prime(x)=0\),即\(3\pi x^{3}-4e = 0\),解得\(x = \sqrt[3]{\frac{4e}{3\pi}}\),设\(x_0 = \sqrt[3]{\frac{4e}{3\pi}}\)。\\
    当\(0 < x < x_0\)时,\(3\pi x^{3}-4e<0\),\(f^\prime(x)<0\),\(f(x)\)单调递减;\\
    当\(x > x_0\)时,\(3\pi x^{3}-4e>0\),\(f^\prime(x)>0\),\(f(x)\)单调递增。

    %步骤四:分析函数\(f(x)\)的极限和极值
    %极限情况
    \(\lim_{x \to 0^{+}}f(x)=\lim_{x \to 0^{+}}(\pi x^{3}-4e\ln x)= +\infty\);\\
    \(\lim_{x \to +\infty}f(x)=\lim_{x \to +\infty}(\pi x^{3}-4e\ln x)= +\infty\)。\\
    %极值情况
    \(f(x)\)在\(x = x_0\)处取得极小值也是最小值\(f(x_0)=\pi x_0^{3}-4e\ln x_0\)。\\
    因为\(3\pi x_0^{3}= 4e\),即\(\pi x_0^{3}=\frac{4e}{3}\),所以\(f(x_0)=\frac{4e}{3}-4e\ln x_0 = \frac{4e}{3}(1 - 3\ln x_0)\)。
    由于\(x_0 = \sqrt[3]{\frac{4e}{3\pi}}<1\),则\(\ln x_0<0\),\(1 - 3\ln x_0>0\),所以\(f(x_0)>0\)。 

    由函数单调性、极值和极限可知,\(f(x)\)的图象与\(x\)轴无交点,即\(f(x)\)在\((0, +\infty)\)内零点个数为\(0\)。所以答案选A。
\end{solution}
\section*{分块矩阵的逆矩阵与伴随矩阵[4-6题]}
%4
\begin{question}
    设  $\mathbf{A}$, $\mathbf{B}$, $\mathbf{C}$  均为 3 阶矩阵,  $\mathbf{A}^{*}$,$\mathbf{B}^{*}$  分别为  $\mathbf{A}$, $\mathbf{B}$  的伴随矩阵,若 $ |\mathbf{A}| = 2$,$|\mathbf{B}| = 3$  ,则分块矩阵  $\left(\begin{array}{ll}\mathbf{C} & \mathbf{A} \\ \mathbf{B} & \mathbf{O}\end{array}\right)  $的伴随矩阵为
    \begin{choices}
      \item $\left(\begin{array}{cc}\mathbf{A}^{*} \mathbf{C B} & 2 \mathbf{A}^{*} \\ 3 \mathbf{B}^{*} & \mathbf{O}\end{array}\right) $ .
      \item $\left(\begin{array}{cc}-\mathbf{A}^{*} \mathbf{C B}^{*} & 2 \mathbf{B}^{*} \\ 3 \mathbf{A}^{*} & \mathbf{O}\end{array}\right)$.
      \item $\left(\begin{array}{cc}\mathbf{O} & -2 \mathbf{A}^{*} \\ -3 \mathbf{B}^{*} & \mathbf{A}^{*} \mathbf{C B}\end{array}\right)$
      \item $\left(\begin{array}{cc}\mathbf{O} & -2 \mathbf{B}^{*} \\ -3 \mathbf{A}^{*} & \mathbf{A}^{*} \mathbf{C} \mathbf{B}^{*}\end{array}\right)$.
    \end{choices}
  \end{question}
\begin{solution}
    设\(\mathbf{A,B}\)为\(n\)阶矩阵,对于分块矩阵\(\mathbf{M=\left(\begin{array}{ll}C&A\\B&O\end{array}\right)}\),有\(\mathbf{\vert M\vert=(-1)^{3\times3}\vert A\vert\vert B\vert=-6}\)。
    
    根据伴随矩阵的性质\(\mathbf{M^{*}=\vert M\vert M^{-1}}\),先求\(\mathbf{M}\)的逆矩阵。
    
    设\(\mathbf{M^{-1}=\left(\begin{array}{cc}X_{1}&X_{2}\\X_{3}&X_{4}\end{array}\right)}\),由\(\mathbf{MM^{-1}=I}\)可得:
    \(\mathbf{\left(\begin{array}{ll}C&A\\B&O\end{array}\right)\left(\begin{array}{ll}X_{1}&X_{2}\\X_{3}&X_{4}\end{array}\right)=\left(\begin{array}{ll}I&O\\O&I\end{array}\right)}\)
    
    即\(\mathbf{\begin{cases}CX_{1}+AX_{3}=I&(1)\\CX_{2}+AX_{4}=O&(2)\\BX_{1}=O&(3)\\BX_{2}=I&(4)\end{cases}}\)
    
    由\((3)\)式得\(\mathbf{X_{1}=B^{-1}O = O}\),由\((4)\)式得\(\mathbf{X_{2}=B^{-1}}\)。
    
    将\(\mathbf{X_{1}=O}\)代入\((1)\)式得\(\mathbf{AX_{3}=I}\),则\(\mathbf{AX_{3}=I}\)。
    
    将\(\mathbf{X_{2}=B^{-1}}\)代入\((2)\)式得\(\mathbf{CX_{2}+AX_{4}=O}\),即\(\mathbf{CB^{-1}+AX_{4}=O}\),\(\mathbf{X_{4}=-A^{-1}CB^{-1}}\)。
    
    所以\(\mathbf{M^{-1}=\left(\begin{array}{cc}O&B^{-1}\\A^{-1}&-A^{-1}CB^{-1}\end{array}\right)}\)
    
    又因为\(\mathbf{A^{*}=\vert A\vert A^{-1}}\),\(\mathbf{B^{*}=\vert B\vert B^{-1}}\),\(\mathbf{\vert M\vert=-6}\),\par 所以\(\mathbf{M^{*}=\vert M\vert M^{-1}=-6\left(\begin{array}{cc}O&B^{-1}\\A^{-1}&-A^{-1}CB^{-1}\end{array}\right)}\)
    
    \(=\left(\begin{array}{cc}\mathbf{O}& - 2\mathbf{B}^{*}\\-3\mathbf{A}^{*}&\mathbf{A}^{*}\mathbf{C}\mathbf{B}^{*}\end{array}\right)\)
    
    答案选D。
\end{solution}
%5 第4题补充习题1
\begin{question}\textbf{[第4题补充习题]}
    设\(\mathbf{P},\mathbf{Q},\mathbf{R}\)均为\(3\)阶矩阵,\(\mathbf{P}^{*},\mathbf{Q}^{*}\)分别为\(\mathbf{P},\mathbf{Q}\)的伴随矩阵,若\(\vert\mathbf{P}\vert = 1\),\(\vert\mathbf{Q}\vert = 4\),则分块矩阵\(\left(\begin{array}{ll}\mathbf{R}&\mathbf{P}\\\mathbf{Q}&\mathbf{O}\end{array}\right)\)的伴随矩阵为( \qquad )
    \begin{choices}
        \item \(\left(\begin{array}{cc}\mathbf{O}& - \mathbf{P}^{*}\\-4\mathbf{Q}^{*}&\mathbf{P}^{*}\mathbf{R}\mathbf{Q}^{*}\end{array}\right)\)
        \item \(\left(\begin{array}{cc}\mathbf{O}& - \mathbf{Q}^{*}\\-4\mathbf{P}^{*}&\mathbf{P}^{*}\mathbf{R}\mathbf{Q}^{*}\end{array}\right)\)
        \item \(\left(\begin{array}{cc}\mathbf{P}^{*}\mathbf{R}\mathbf{Q}^{*}& - \mathbf{P}^{*}\\-4\mathbf{Q}^{*}&\mathbf{O}\end{array}\right)\)
        \item \(\left(\begin{array}{cc}\mathbf{P}^{*}\mathbf{R}\mathbf{Q}^{*}& - 4\mathbf{Q}^{*}\\- \mathbf{P}^{*}&\mathbf{O}\end{array}\right)\)
    \end{choices}
\end{question}
\begin{solution}
    做法和第4题一样,选B。
\end{solution}
%6 第4题补充习题2
\begin{question}
    设\(\mathbf{A},\bm{B},\mathbf{C}\)均为\(4\)阶矩阵,\(\mathbf{A}^{*},\mathbf{B}^{*}\)分别为\(\mathbf{A},\mathbf{B}\)的伴随矩阵,若\(\vert\mathbf{A}\vert = 3\),\(\vert\mathbf{B}\vert = 2\),则分块矩阵\(\left(\begin{array}{ll}\mathbf{C}&\mathbf{A}\\\mathbf{B}&\mathbf{O}\end{array}\right)\)的伴随矩阵为(\qquad )
    \begin{choices}
        \item \(\left(\begin{array}{cc}\mathbf{O}& 3\mathbf{A}^{*}\\-2\mathbf{B}^{*}&\mathbf{A}^{*}\mathbf{C}\mathbf{B}^{*}\end{array}\right)\)
        \item \(\left(\begin{array}{cc}\mathbf{O}& 3\mathbf{B}^{*}\\2\mathbf{A}^{*}&-\mathbf{A}^{*}\mathbf{C}\mathbf{B}^{*}\end{array}\right)\)
        \item \(\left(\begin{array}{cc}\mathbf{A}^{*}\mathbf{C}\mathbf{B}^{*}& - 3\mathbf{A}^{*}\\-2\mathbf{B}^{*}&\mathbf{O}\end{array}\right)\)
        \item \(\left(\begin{array}{cc}\mathbf{A}^{*}\mathbf{C}\mathbf{B}^{*}& - 2\mathbf{B}^{*}\\-3\mathbf{A}^{*}&\mathbf{O}\end{array}\right)\)
    \end{choices}
\end{question}
\begin{solution}
    此题同第4题的做法,选B
\end{solution}
\section*{正定二次型及其矩阵[7-8题]}
%7 
\begin{question}
    设三元二次型 $ f=\mathbf{x}^{\mathrm{T}} \mathbf{A} \mathbf{x} $ 正定,其中  $\mathbf{x}=\left(x_{1}, x_{2}, x_{3}\right)^{\mathrm{T}}, \mathbf{A}$ 为实对称矩阵,则下列说法不正确的是
    \begin{choices}
      \item 仅在  $\mathbf{x}=0 $ 处 $ f$  取得最小值.
      \item 齐次线性方程组  $\begin{cases}\frac{\partial f}{\partial x_{1}}=0, \\[5pt] \frac{\partial f}{\partial x_{2}}=0,  \\[5pt] \frac{\partial f}{\partial x_{3}}=0\end{cases}$只有零解. 
      \item $f $ 的二阶偏导数矩阵  $\left(\frac{\partial^{2} f}{\partial x_{i} \partial x_{j}}\right)_{3 \times 3}=\left(\begin{array}{ccc}\frac{\partial^{2} f}{\partial x_{1}^{2}} & \frac{\partial^{2} f}{\partial x_{1} \partial x_{2}} & \frac{\partial^{2} f}{\partial x_{1} \partial x_{3}} \\ \frac{\partial^{2} f}{\partial x_{2} \partial x_{1}} & \frac{\partial^{2} f}{\partial x_{2}^{2}} & \frac{\partial^{2} f}{\partial x_{2} \partial x_{3}} \\ \frac{\partial^{2} f}{\partial x_{3} \partial x_{1}} & \frac{\partial^{2} f}{\partial x_{3} \partial x_{2}} & \frac{\partial^{2} f}{\partial x_{3}^{2}}\end{array}\right) $ 正定.
      \item 存在$ 3$ 维非零列向量$\mathbf{\alpha}$  ,使得 $ \mathbf{A}=\mathbf{\alpha} \mathbf{\alpha}^{\mathrm{T}}$  ,从而  $f=\left(\mathbf{\alpha}^{\mathrm{T}} \mathbf{x}\right)^{2}$  .
    \end{choices}
\end{question}
\begin{solution}
    由于\(f = \mathbf{X}^T \mathbf{A} \mathbf{X}\)正定,所以对于任意\(x \neq 0\),\(f>0
    = f\big|_{x=0} \),\(A\)正确\par
    因为\(\begin{cases}
        \frac{\partial f}{\partial x_1} = 0,\\
        \frac{\partial f}{\partial x_2} = 0,\\
        \frac{\partial f}{\partial x_3} = 0,
    \end{cases}\)即\(2 \mathbf{Ax = 0}\),且\(
        |\mathbf{A}| >0
   \),所以\(  \begin{cases}
 
        \frac{\partial f}{\partial x_1} = 0,\\
        \frac{\partial f}{\partial x_2} = 0,\\
        \frac{\partial f}{\partial x_3} = 0,
    \end{cases}\)只有零解\par
    因为\(\left( \frac{\partial^2 f}{\partial x_i \partial x_j}\right)_{3\times3} = 2 \mathbf{A} \)
    ,所以\(C\)正确\par
    \(D\)不正确。因为若存在$ 3$ 维非零列向量$\mathbf{\alpha}$,使得 $ \mathbf{A}=\mathbf{\alpha} \mathbf{\alpha}^{\mathrm{T}}$  
    则\(r(\mathbf{A}) \leq r(\mathbf{\alpha}) = 1\),这与\(\mathbf{A}\)是正定矩阵相矛盾
\end{solution}
%8 第7题补充1
\begin{question}
    设三元二次型 \( f = \mathbf{x}^T\mathbf{B}\mathbf{x} \) 正定,\( \mathbf{x} = (x_1, x_2, x_3)^T \),\( \mathbf{B} \) 为实对称矩阵,下列说法错误的是(\quad)
    \begin{choices}
        \item \( f \) 在 \( \mathbf{x} = 0 \) 处取得最小值。
        \item 齐次线性方程组 
        \(
        \begin{cases}
        \frac{\partial f}{\partial x_1} = 0 \\
        \frac{\partial f}{\partial x_2} = 0 \\
        \frac{\partial f}{\partial x_3} = 0
        \end{cases}
        \)只有零解。 
        \item \( f \) 的二阶偏导数矩阵正定。
        \item 对任意三维非零列向量 \( \bm{\beta} \),\( \mathbf{B} + \bm{\beta}\bm{\beta}^T \) 负定。
    \end{choices}
\end{question}
\begin{solution}
    A:正定二次型在 \(\mathbf{x}=0\) 处取最小值,正确.\par B:求偏导后得 \(2\mathbf{B}\mathbf{x}=0\),\(\mathbf{B}\) 正定可逆,只有零解,正确。
    \par C:二阶偏导矩阵为 \(2\mathbf{B}\),正定,正确。\par
    D:\(\mathbf{B}\) 正定,\({\beta} {\beta}^T\) 半正定,\(\mathbf{B} + {\beta} {\beta}^T\) 正定,D 错误。
\end{solution}

\section*{相似矩阵的行列式和迹相等[9-11题]}
%9 第9题
\begin{question} 
    已知A为3阶矩阵,\(\bm{\alpha}_1,\bm{\alpha}_2,\bm{\alpha}_3\)是线性无关的3维列向量,且\((\mathbf{A - E})\bm{\alpha}_1= + \bm{\alpha}_2+\bm{\alpha}_3\),\((\mathbf{A - E})\bm{\alpha}_2=\bm{\alpha}_1+\bm{\alpha}_2 + 4 \bm{\alpha}_3\),\((\mathbf{A - E})\bm{\alpha}_3=\bm{\alpha}_1+3 \bm{\alpha}_2 + 8 \bm{\alpha}_3\),则\(\vert \bm{A}\vert=\rule[-0.5ex]{3cm}{0.4pt}\) 。
\end{question}
\begin{solution}
    \(\bm{A}\bm{\alpha}_1=\bm{\alpha}_1 + \bm{\alpha}_2 +\bm{\alpha}_3 , \bm{A} \bm{\alpha}_2 = \bm{\alpha}_1 + 2 \bm{\alpha}_2
    +4 \bm{\alpha}_3, \bm{A}\bm{\alpha}_3 = \bm{\alpha}_1 + 3 \bm{\alpha}_2 + 9 \bm{\alpha}_3\) \vspace{5pt}\par
    所以\begin{align*}&\vert \bm{A} \vert \cdot \vert \bm{\alpha}_1, \bm{\alpha}_2, \bm{\alpha}_3 \vert = \vert \bm{\alpha}_1, \bm{\alpha}_2, \bm{\alpha}_3\vert \cdot \begin{vmatrix}
        1&1&1\\
        1&2&3\\
        1&4&9
    \end{vmatrix} \\
    \end{align*}
    故\(\vert \bm{A} \vert= \begin{vmatrix}
        1&1&1\\
        1&2&3\\
        1&4&9
    \end{vmatrix} = \begin{vmatrix}
        1&1&1\\
        0&1&2\\
        0&3&8
    \end{vmatrix} = 2\)
\end{solution}

%10 第10题
\begin{question}
    已知$\bm{A}$为三阶矩阵,$\bm{\alpha}_1,\bm{\alpha}_2,\bm{\alpha}_3$是线性无关的3维列向量,满足:
    \begin{equation*}
        \begin{cases}  % 使用紧凑型 cases 环境
            (\bm{A}+2 \bm{E}) \bm{\alpha}_1 = \bm{\alpha}_2 + \bm{\alpha}_3, \\ 
            (\bm{A}+2 \bm{E}) \bm{\alpha}_2 = \bm{\alpha}_1 + 3 \bm{\alpha}_3, \\ 
            (\bm{A}+2 \bm{E}) \bm{\alpha}_3 = 2 \bm{\alpha}_1 + \bm{\alpha}_3 
        \end{cases} 
    \end{equation*}
    求$\vert\bm{A}\vert$ 
\end{question}

\begin{solution}
    \textbf{法1}
    
    根据已知条件得:
    \begin{align*}
        \bm{A} (\bm{\alpha}_1, \bm{\alpha}_2,\bm{\alpha}_3) 
        &= (\bm{\alpha}_1, \bm{\alpha}_2,\bm{\alpha}_3)
        \begin{pmatrix}
            -2 & 1  & 1  \\
            1  & -2 & 3  \\
            2  & 0  & -1
        \end{pmatrix} \\
        \Rightarrow \bm{A} &\sim \bm{P}, \quad \text{其中 } \bm{P} = 
        \begin{pmatrix}
            -2 & 1  & 1  \\
            1  & -2 & 3  \\
            2  & 0  & -1
        \end{pmatrix} \\
        \vert\bm{A}\vert &= \begin{vmatrix}
            -2 & 1  & 1  \\
            1  & -2 & 3  \\
            2  & 0  & -1
        \end{vmatrix} \\
        &= \begin{vmatrix}
            0  & 1  & 1  \\
            7  & -2 & 3  \\
            0  & 0  & -1
        \end{vmatrix} = -1 \cdot (-7) = 7.
    \end{align*}
\newpage
    \textbf{法2}
    
    根据已知条件得:
    \begin{align*}
        &(\bm{A} +2\bm{E}) (\bm{\alpha}_1, \bm{\alpha}_2,\bm{\alpha}_3) 
        = (\bm{\alpha}_1, \bm{\alpha}_2,\bm{\alpha}_3)
        \begin{pmatrix}
            0 & 1  & 2  \\
            1  & 0 & 0  \\
            1  & 3  & 1
        \end{pmatrix} \\
        &\Rightarrow \bm{A} + 2 \bm{E} \sim \bm{P}, \quad \text{其中 } \bm{P} = 
        \begin{pmatrix}
            0 & 1  & 2  \\
            1  & 0 & 0  \\
            1  & 3  & 1
        \end{pmatrix} \\
        &\Rightarrow\bm{A} \sim \bm{P}- 2\bm{E} = \begin{pmatrix}
            -2 & 1  & 2 \\
            1  & -2 & 0 \\
            1  & 3 & -1
        \end{pmatrix} \\
        &\Rightarrow \vert \bm{A} \vert= \begin{vmatrix}
            -2 & 1  & 2 \\
            1  & -2 & 0 \\
            1  & 3 & -1
        \end{vmatrix} =\begin{vmatrix}
            -2 & -3  & 2 \\
            1  & 0 & 0 \\
            1  & 5 & -1
        \end{vmatrix}= (-1)^{2+1}[(-1)\times(-3) -2 \times 5] = 7
    \end{align*}
\end{solution}
%11 第11题
\begin{question}
    已知$\bm{A}$为二阶矩阵,$\bm{\beta}_1,\bm{\beta}_2$是线性无关的2维列向量,且满足:
    \begin{equation*}
        \begin{cases}  % 使用紧凑型 cases 环境
            (\bm{A}-3 \bm{E}) \bm{\beta}_1 = 2 \bm{\beta}_1 +\bm{\beta}_2 , \\ 
            (\bm{A}-3 \bm{E}) \bm{\beta}_2 =  4 \bm{\beta}_1 - \bm{\beta}_2 ,  
        \end{cases} 
    \end{equation*}
    求\(\bm{A}\)的迹\(tr(\bm{A})\)和行列式$\vert\bm{A}\vert$ 。
\end{question}
\section*{切比雪夫不等式[12-16题]}
%12题
\begin{question}
    设随机变量 $X$ 的概率密度函数为偶函数,方差 $D(X) = 1$。若已知用切比雪夫不等式估计得 $P(|X| < \epsilon) \geq 0.96$,则常数 $\epsilon = \rule[-0.5ex]{3cm}{0.4pt}$。
\end{question}
\begin{solution}
    由于随机变量\(\bm{X}\)的概率密度是偶函数,故函数\[E(\bm{X}) = \int_{-\infty}^{+\infty} xf(x) \,dx = 0\]
    根据切比雪夫不等式得,\[P\{\vert \bm{X}  \vert \geq \epsilon\} \leq \frac{1}{\epsilon^2}\]
    根据对立事件的性质得,\[P\{\vert \bm{X}  \vert < \epsilon\} \geq 1-\frac{1}{\epsilon^2}\]
    结合题目条件得,\[1-\frac{1}{\epsilon^2} = 0.96\]
    解得 \(\epsilon  = 5\)
\end{solution}
%13题
\begin{question}
    设随机变量 $X$ 的概率密度函数为偶函数,方差 $D(X) = 16$。若已知用切比雪夫不等式估计得 $P(|X| < \epsilon) \geq 0.75$,则常数 $\epsilon = \rule[-0.5ex]{3cm}{0.4pt}$。
\end{question}
\begin{solution}
    由于随机变量\(\bm{X}\)的概率密度是偶函数,故函数\[E(\bm{X}) = \int_{-\infty}^{+\infty} xf(x) \,dx = 0\]
    根据切比雪夫不等式得,\[P\{\vert \bm{X}  \vert \geq \epsilon\} \leq \frac{16}{\epsilon^2}\]
    根据对立事件的性质得,\[P\{\vert \bm{X}  \vert < \epsilon\} \geq 1-\frac{16}{\epsilon^2}\]
    结合题目条件得,\[1-\frac{1}{\epsilon^2} = 0.96\]
    解得 \(\epsilon  = 8\)
\end{solution}
%14题
\begin{question}
    设随机变量 $X$ 的概率密度函数为偶函数,方差 $D(X) = 0.25$。若已知用切比雪夫不等式估计得 $P(|X| < \epsilon) \geq 0.99$,则常数 $\epsilon = \rule[-0.5ex]{3cm}{0.4pt}$。
\end{question}
\begin{solution}
    由于随机变量\(\bm{X}\)的概率密度是偶函数,故函数\[E(\bm{X}) = \int_{-\infty}^{+\infty} xf(x) \,dx = 0\]
    根据切比雪夫不等式得,\[P\{\vert \bm{X}  \vert \geq \epsilon\} \leq \frac{0.25}{\epsilon^2}\]
    根据对立事件的性质得,\[P\{\vert \bm{X}  \vert < \epsilon\} \geq 1-\frac{0.25}{\epsilon^2}\]
    结合题目条件得,\[1-\frac{1}{\epsilon^2} = 0.99\]
    解得 \(\epsilon  = 5\)
\end{solution}
%15题 结合其他考点
\begin{question}
    设随机变量 \(\bm{X}\)与\(\bm{Y}\)相互独立,其中\(\bm{X}\)的概率密度为偶函数,
    方差\(D(\bm{X}) = 4\),\(Y\)服从参数\(\lambda =1\)的指数分布。定义\(\bm{Z} = \bm{X}+\bm{Y}\),利用切比雪夫不等式
    估计\(P(\vert \bm{Z}-E(\bm{Z})\vert \geq \epsilon) \leq 0.96\),求\(\epsilon\)
    的最小值。
\end{question}
\begin{solution}
    因为随机变量\(\bm{X}\)与\(\bm{Y}\)相互独立,故
    \[E(\bm{Z}) =E(\bm{X}) +E(\bm{Y})=1\]
    \[D(\bm{Z}) =D(\bm{Z})+D(\bm{Z})=4+1=5\]
    根据切比雪夫不等式得,
    \[P(\vert \bm{Z}- 1\vert \geq \epsilon) \leq \frac{5}{\epsilon^2}\]
    根据独立事件的性质得,
    \[P(\vert \bm{Z}- 1\vert < \epsilon) \geq 1-\frac{5}{\epsilon^2}\]
    结合题目条件得,
    \[1 - \frac{5}{\epsilon^2} \geq 0.96\]
    解得\(\epsilon \geq 5\sqrt{5}\)\\
    故\(\epsilon\)最小值为\(5\sqrt{5}\)
\end{solution}
%16题 结合其他考点
\begin{question}
    设随机变量 \(\bm{X}\)与\(\bm{Y}\)相互独立,其中\(\bm{X}\)与\(\bm{Y}\)的概率密度为偶函数,
    方差\(D(\bm{X}) =D(\bm{Y}) = 1\)。定义\(\bm{Z} = \bm{X}+\bm{Y}\),利用切比雪夫不等式
    估计\(P(\vert \bm{Z}-E(\bm{Z})\vert \geq \epsilon )\geq 0.88\),求\(\epsilon\)
    的最小值。
\end{question}
\begin{solution}
    因为随机变量\(\bm{X}\)与\(\bm{Y}\)相互独立,故
    \[E(\bm{Z}) =E(\bm{X}) +E(\bm{Y})=0\]
    \[D(\bm{Z}) =D(\bm{Z})+D(\bm{Z})=2\]
    根据切比雪夫不等式得,
    \[P(\vert \bm{Z}\vert \geq \epsilon) \leq \frac{2}{\epsilon^2}\]
    根据独立事件的性质得,
    \[P(\vert \bm{Z}- 1\vert < \epsilon) \geq 1-\frac{2}{\epsilon^2}\]
    结合题目条件得,
    \[1 - \frac{2}{\epsilon^2} \geq 0.88\]
    解得\(\epsilon \geq \frac{5\sqrt{6}}{3}\)\\
    故\(\epsilon\)最小值为\(\frac{5\sqrt{6}}{3}\)
\end{solution}
\section*{隐函数存在定理[17-21题]}
%17
\begin{question}
    若函数 \( y = y(x) \) 由方程 \( x^{3} + y^{3} + xy - 1 = 0 \) 确定,求 \( \lim_{x \to 0} \frac{3y + x - 3}{x^{3}} \)。
\end{question}
\begin{solution}
    设\(F(x,y) = x^3 +y^3 + xy -1\),则
    \[F_{x}^{\prime}(x,y) = 3x^2+y,F_{y}^{\prime} (x,y) = 3y^2+x\]
    根据隐函数存在定理得,\(y^{\prime}(x) =\frac{F_{x}^{\prime}}{F_{y}^{\prime}} 
    = \frac{3x^2+y}{3y^2+x} (3y^2+x \neq 0)\)\\设\(g(x) = 3x^2+y,h(x)=3y^2+x\),
    则\(y^{\prime\prime}(x) = \frac{h^{\prime}(x)g(x)-h_(x)g^{\prime}(x)}{(h(x))^2} (h(x)\neq0)\)\\
    设\(p(x) =h^{\prime}(x)g(x)-h_(x)^{\prime}(x),q(x) = (h(x))^2 \)
    \[y^{\prime\prime}(x) = \frac{p(x)}{q(x)}\]
    而\( y^{\prime\prime\prime}(x) = \frac{p^\prime(x)q(x)-p(x)q^\prime(x)}{[q(x)]^2}\),其中\(3y^2+x\neq 0\)\\
    带入\(x=0\),得\(y=1\),进而\(y^\prime(x) = -\frac{1}{3}\)\\
    带入\(y^\prime(0) = -\frac{1}{3}\),得\(y^{\prime\prime} (0)= 0\)\\
    经计算得\(y^{\prime\prime\prime}(0)=-\frac{56}{27}\)\\
    连续使用三次洛必达法则,得原极限\(=\frac{1}{2}\lim_{x\to0}y^{\prime\prime\prime}(x)=-\frac{26}{27}\)
\end{solution}
%18
\begin{question}
    设函数\(y = y(x)\) 由方程\(x^3+y^3 + \int_{0}^{y}cost^2 \,dt -1= 0\)确定,求\(\lim\limits_{x\to 1} \frac{y+3x-3}{(x-1)^2}\)。
\end{question}
\begin{solution}
    将\(x=1\)带入原方程,得
    \[\]
    \begin{equation}
        1-1+y^3+\int_{0}^{y} cost^2\,dt = 0
    \end{equation}
    设\(g(y) = y^3+\int_{0}^{y} cost^2\,dt \),则\(g^{\prime}(y)= 3y^2 +cosy^2 \geq 0\)\\
    于是函数\(g(y)\)在\((-\infty,+\infty)\)内单调递增\\
    而\(g(0) = 0\),所以\(y=0\)是方程\((1)\)的唯一解
    原方程两边guanyu\(x\)求导得,
    \begin{equation}
        3x^2 +3y^2\cdot y^\prime +cosy^2 \cdot y^\prime= 0
    \end{equation}
    将\(x=1,y=0\)带入方程(2)得,
    \[3 \cdot 1^2 +3 \cdot 0^2 \cdot y^\prime +cos0^2 \cdot y^\prime =0\]
    即 \(y^\prime =-3\)\\
    方程\((2)\)两边关于\(x\)求导得,
    \begin{equation}
        6x+6y\cdot(y^\prime)^2+3y^2\cdot y^{\prime\prime}+cosy^2 \cdot y^{\prime\prime} -2y \cdot y^\prime siny^2=0
    \end{equation}
    将\(x=1,y=0,y^\prime =-3\)带入方程\((3)\)得,
    \[6\cdot1 + y^{\prime\prime}=0\]
    解得\(y^{\prime\prime} = -6\)\\
    对原极限使用两次洛必达法则,得\(I = \frac{1}{2}y^{\prime\prime}(1) = -3\)
\end{solution}
%19
\begin{question}
    设函数 $ y = y(x) $ 由方程
$$
x^3 + y^3 + \int_{0}^{y} \cos t^2 \, dt = 1
$$
    确定,且满足微分方程 $ y'' + y' = 1 $,求 $ y(x) $。
\end{question}
\begin{solution}
    \(x=1,y=0\)是满足原方程的一组解\\
    对原方程两边关于\(x\)求导,得
    \begin{equation}
        3x^2+3y^2y^{\prime}+y^\prime cosy^2 =0
    \end{equation}
    将\(x=1,y=0\)带入方程\((1)\)得\(y^\prime(1) =-3 \)
    微分方程的特征方程为\[\lambda^2+\lambda =0\]
    解得\(\lambda_1=0,\lambda_2=-1\)\\
    故通解为\(y=C_1+C_2e^{-x}\),\(y^\prime = -C_2e^{-x}\)
    代入初始条件得,
    \[\begin{cases}
        C_1+C_2e=0\\
        -C_2e^{-1}=-3
    \end{cases}\]
    解得\[\begin{cases}
        C_1 = -3e^2\\
        C_2=3e
    \end{cases}\]
    故\(y=3\cdot e^{1-x} - 3 e^2\)
\end{solution}
%20
\begin{question}
    设函数 $ y = y(x) $ 由方程
$$
x^3 + y^3 + \int_{0}^{y} \cos t^2 \, dt = 1
$$
确定,求 $ y(x) $ 在 $ x = 1 $ 处的二阶泰勒展开式。
\end{question}
\begin{solution}
    把\(x\)带入原方程得,\[1+y^3+\int_{0}^{y}cost^2\,dt = 1\]
    把\(y=0\)带入上式,方程成立,并且容易检验\(y=0\)是该方程的唯一解\\
    题目原方程两边关于\(x\)求导得,
    \begin{equation}
        3x^2 + 3 y^2 y^\prime + cosy^2 \cdot y^\prime = 0
    \end{equation}
    把\(x=1,y=0\)带入方程(1)得,\(y^\prime(1) = -3\)\\
    方程\((1)\)两边关于\(x\)求导得,
    \begin{equation}
        6x + 6 y [y^\prime]^2 + 3y^2y^{\prime\prime} + y^{\prime\prime}cosy^2 -2ysiny^2(y^\prime)^2 = 0
    \end{equation}
    将\(x =1,y=0,y^\prime(1) = -3\)带入方程\((2)\)得\[6+y^{\prime\prime} =0\]
    解得\(y^{\prime\prime}(1) = -6\)\\
    故\(y(x)\)在\(x=1\)处的二阶泰勒展开式为\(y(x) = -3(x-1) -3(x-1)^2 +o[(x-1)^2]\)
\end{solution}
%21
\begin{question}
    设函数\(y = y(x)\)由方程\(x^{3} + y^{3} + \int_{0}^{y} \cos t^{2} dt = 1\)确定,且\(x = t^{2}\),求\(\frac{dy}{dt}\)在\(t = 1\)处的值。
\end{question}
\begin{solution}
    首先,对方程 \(x^{3} + y^{3} + \int_{0}^{y} \cos t^{2} dt = 1\) 两边关于 x 求导:\[3x^{2} + 3y^{2} \frac{dy}{dx} + \cos y^{2} \cdot \frac{dy}{dx} = 0\]
    解出 \(\frac{dy}{dx}\):\(\frac{dy}{dx} = -\frac{3x^{2}}{3y^{2} + \cos y^{2}}\)\\
    已知 \(x = t^{2}\),则 \(\frac{dx}{dt} = 2t\)。\\
    根据链式法则 \(\frac{dy}{dt} = \frac{dy}{dx} \cdot \frac{dx}{dt}\)。\\
    当 \(t = 1\) 时,\(x = 1^{2} = 1\)。\\
    将 \(x = 1\) 代入原方程,得:\(1^{3} + y^{3} + \int_{0}^{y} \cos t^{2} dt = 1 \implies y^{3} + \int_{0}^{y} \cos t^{2} dt = 0\)
    显然 \(y = 0\) 满足该方程。
    将 \(x = 1\),\(y = 0\) 代入 \(\frac{dy}{dx}\):\(\frac{dy}{dx} \Big|_{x=1, y=0} = -\frac{3 \cdot 1^{2}}{3 \cdot 0^{2} + \cos 0^{2}} = -3\)\\
    又 \(\frac{dx}{dt} \big|_{t=1} = 2 \cdot 1 = 2\),因此:\(\frac{dy}{dt} \Big|_{t=1} = \frac{dy}{dx} \cdot \frac{dx}{dt} \Big|_{t=1} = -3 \cdot 2 = \boxed{-6} \)
\end{solution}
\section*{条件极值[22-26]}
%22
\begin{question}
    求函数 \(f(x,y) = (x^{2} + y^{2})e^{-(x+y)}\) 在区域 \(D = \{(x,y) \mid x \geqslant 0, y \geqslant 0, x^{2} + y^{2} \leqslant 16\}\) 上的最大值及最小值。
\end{question}
\begin{solution}
    设直线段\(L_1 = \{(x,y)|x = 0,0\leq y \leq 4\}\),直线段\(L_2={(x,y)|y=0,0\leq x\leq 4}\)\\
    曲线段\(L_3 = \{(x,y)|x^2+y^2=16,x\geq 0,y\geq 0\}\),\\
    开区域\(\Omega = {\{(x,y)|x^2+y^2<16,x>0,y>0\}}\)\\
    当点\(P(x,y) \in L_1\)时,\(f(x,y) =g(y)= y^2e^{-y} (0\leq y \leq 4)\)\\
    而\(g^\prime(y) = - (y-2)ye^{-y} \)\\
    \(\implies g^\prime(y)>0,y\in(0,2)\text{且}g^\prime(y)<0,y\in(2,4)\)\\
    \(\implies g(y)\text{在}(0,2)\text{内}\)单调递增,在\((2,4)\)内单调递减\\
    \(f(x,y)\)在直线段\(L_1\)内的最大值为\(4e^{-4}\),最小值为\(0\)\\
    当点\(P(x,y) \in L_2\)时,有对称性得\(f(x,y)\)的最大值与最小值与\(L_1\)内一致\\
    当点\(P(x,y) \in \Omega\)时,令偏导数得零,有如下方程组:
    \begin{equation}
        \begin{cases}
            f_{x}^{\prime}(x,y) = (2x -x^2 -y^2) e^{-x-y}&= 0\\
            f_{y}^{\prime}(x,y) = (2y - y^2 -x^2)e^{-y-x}&=0
        \end{cases}
    \end{equation}
    解得\(f(x,y)\)在\(\Omega\)内的驻点为\((1,1) \text{且}f(1,1) = 2e^{-2}\)\\
    当\(P(x,y) \in L_3\)时,\(f(x,y) = 16e^{-(x+y)}\),\(f(x,y)\)取得最大值和最小值的点就是
    \((x+y)\)取得最小值和最大值的点。问题可转化为\(h(x,y) = x+y\)在约束条件\(x^2+y^2=16,x\geq0,y\geq0\)的条件极值问题\\
    令\(L(x,y,\lambda) = x+y+\lambda(x^2+y^2-16)\),由\(\begin{cases}
        L_{x}^{\prime} = 1+2\lambda x =0,\\
        L_{y}^{\prime} = 1 +2\lambda y=0,\\
        L_{\lambda}^{\prime} = x^2 +y^2 -16 =0,
    \end{cases}\)得驻点\(x=y=2\sqrt{2}\),由\\
    \(f(0,4) = f(4,0)=16e^-4,f(2\sqrt{2},2\sqrt{2}) = 16e^{-4\sqrt{2}}\)\\
    可知函数\(f(x,y)\)在\(L_3\)上的最大值和最小值分别为\(16e^{-4},16e^{-4\sqrt{2}}\)\\
    比较值 \(0,2e^{-2},4e^{-2}\)的大小,可知函数\(f(x,y)\)在区域\(D\)上的最大值和最小值分别为
    \(4e^{-2}\text{和}0\)
\end{solution}


